%% LyX 2.4.0~RC3 created this file.  For more info, see https://www.lyx.org/.
%% Do not edit unless you really know what you are doing.
\documentclass[british]{article}
\usepackage[T1]{fontenc}
\usepackage[utf8]{inputenc}
\usepackage{babel}
\begin{document}
\title{Artificial intelligence and Turing machines - Part 2: Can machines
think?}
\author{Marco Paladini}

\maketitle
In part 1 of this article, we have seen an introduction to the concept
of Turing machine, and some examples of why they're useful in studying
the theory of computation. In this part we compare the kind of tasks
that machines can do with the tasks that we humans do, and ask the
question, can a machine think?

\emph{If we believe that our bodies and our minds perform some kind
of computation when we receive information from the outside world
through our five senses (sight, sound, touch, taste, smell), then
we can be confident in saying that we are machines too}. Let me clarify.
Of course we’re not machines in the sense that we respond to a given
input with a given output, we’re much more complex than that.

\emph{But even simple machines can give rise to extraordinary complexity,
and to convince you, I’ll give you a few examples.}

\section*{Simple machines, complex behaviour}

Pseudo-random number generators {[}1{]} are algorithms following a
very precise set of instructions that produce an unpredictable output.
When you run a pseudo-random number generator, you know exactly what
the algorithm is doing, but you can’t predict what the next value
is going to come out in the output.

There is a brilliant test of our own ability to generate random actions
at this website, I urge you to try it if you haven’t yet: http://people.ischool.berkeley.edu/\textasciitilde nick/aaronson-oracle/index.html.
The computer will predict which of two keys you will press on the
keyboard, and will get it right most of the time.

Conway’s game of life {[}2{]} is another typical example of how simple
rules can result in fantastic complexity. Imagine a large board divided
into squares with stones on some of the squares, the rules are to
simply remove stones or add stones according to how many neighbours
they have on the board.

Would you imagine that from such a simple setup we can end up seeing
patterns that look like creatures moving around the board or spaceships
that generate other creatures? The animations on the Wikipedia page
do a good job of showing how the resulting patterns are really life-like.

\section*{Software in general is really complex}

Even machines governed by simple rules can produce complex and difficult
to predict outputs. This is part of the reason why we don’t have a
general way to verify our software. That doesn’t mean we should give
up software verification, in some cases it’s very important and there
might be a way to verify your specific software for a particular application,
but for example, we can't run an automatic analysis on \emph{any}
database program and make sure it only allows authorised users to
read and write to the database. \emph{Next time you think how self-driving
cars and AI programs need to be correct all the time, spare a thought
for all the software you use today, and remember it will have some
still undiscovered bugs and vulnerabilities that can cause it to malfunction}. 

\section*{Are we machines?}

We’ve seen how simple machines can produce complex and unpredictable
computations. Now we ask the question again. Are we machines ourselves?
We certainly are capable of complex and unpredictable computations.

There is no robot technology in the world today with capabilities
similar to human muscles, skin and eyes {[}3{]}, but imagine things
where humans and machine compete on equal footing (e.g. chess playing,
driving a car, programming computers, writing music). Are we sure
there are things we do while driving, that cannot be done by a machine?
Is there some thinking we do while programming, that is fundamentally
un-computable?

If the answer is yes, then are we able to point out exactly what that
is? We should be able to say \textquotedbl a self-driving car cannot
compete with humans because task X is not computable by a machine\textquotedbl .
As far as we are aware, we don't have proof that computers cannot
do what we do. This means that \emph{in theory}, all our thinking
could be simulated by a machine.

\emph{Does it mean we can actually simulate people? No, it doesn’t,
because computation speed, energy consumption, process noise, and
physical architecture matters a lot in practice, but it’s something
we conveniently ignore, when we develop theories}.

\section*{What about Human-level AI (AGI)?}

I hope I convinced you that \emph{in theory} it is possible to build
a machine that has human-level intelligence, does it mean Artificial
General Intelligence (AGI) is just around the corner? Probably not,
because we don't yet know how to do it. I leave you with two quotes
on that. The first one by \emph{Rodney Brooks}, when writing about
the history of AI {[}4{]} he says ”So, journalists, don’t you dare,
don’t you dare, come back to me in ten years and say \emph{where is
that Artificial General Intelligence that we were promised? It isn’t
coming any time soon}.”

The other quote is by \emph{Andrew Ng}, who famously said “\emph{Worrying
about evil AI killer robots today is a little bit like worrying about
overpopulation on the planet Mars}.” {[}5{]}.

\section*{Are machines intelligent?}

It could also be that we will always consider humans to be more intelligent
than machines, no matter how advanced they might be.

After all, we were quick to reconsider the game of chess, from an
activity that required human intelligence, to something easy for computers
to do, that requires no intelligence at all.

Look around, \emph{the world is full of machines that have surpassed
human capabilities}, from fast calculators, large databases of documents,
tall construction cranes lifting tonnes of materials, fast industrial
robots, high-speed racing cars, underground trains... and airplanes,
allowing people to fly in the sky.

{[}1{]} https://en.wikipedia.org/wiki/Pseudorandomness

{[}2{]} https://en.wikipedia.org/wiki/Conway\%27s\_Game\_of\_Life

{[}3{]} If you do have robotic muscles with the energy efficiency
and strength and resiliency of human muscles, let me know. Equally
for robotic skin and robotic eyes.

{[}4{]} https://rodneybrooks.com/forai-the-origins-of-artificial-intelligence/

{[}5{]} https://www.gsb.stanford.edu/insights/andrew-ng-why-ai-new-electricity


\end{document}
