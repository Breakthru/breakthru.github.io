%% LyX 2.4.0~RC3 created this file.  For more info, see https://www.lyx.org/.
%% Do not edit unless you really know what you are doing.
\documentclass[british]{article}
\usepackage[T1]{fontenc}
\usepackage[utf8]{inputenc}
\usepackage{babel}
\begin{document}
\title{My own 3 laws of robotics }
\author{Marco Paladini}

\maketitle
The 3 laws of robotics written by Isaac Asimov in his books are very
famous and are fundamental to his science fiction stories.

\section*{Asimov three laws}

1) A robot may not injure a human being or, through inaction, allow
a human being to come to harm.

2) A robot must obey orders given it by human beings except where
such orders would conflict with the First Law.

3) A robot must protect its own existence as long as such protection
does not conflict with the First or Second Law.

see https://xkcd.com/1613/

\subsection*{Are there any laws of robotics in the real world?}

I'm talking about guidelines for us, the people who build robots,
we all know that robots obey the laws of physics and will do whatever
their software will say{[}1{]}.

So here's my laws of robotics:
\begin{enumerate}
\item Demo setups always break
\item Videos don't tell the whole story
\item Robotics is expensive
\end{enumerate}

\subsection*{1) Demo setups always break}

This is the reality of setting up a robot demo to showcase your robotics
results.

\emph{Make sure your demo is constantly in tip-top shape}, there are
plenty of things that go wrong from one day to the next: cables that
snap, parts that break: unexpected obstacles, objects or containers
in the wrong place, etc. etc.

Unless you're doing pure research, making sure your robot demo can
run every day and does not break easily (and you don't actually leave
it broken) is the one proof that your product will work tomorrow.

\subsection*{2) Videos don't tell the whole story}

\emph{We all know robotics is hard}. People see the one impressive
robot demo video and the imagination wanders to other amazing things
that this robot can do because now that it has performed that task
this once in this video the robot has \textquotedbl mastered this
task\textquotedbl .

Nothing can be further from the truth. One impressive robot demo video
does not mean the robot will work day-in-day-out, only actually working
in production at the required reliability will demonstrate that.

\emph{We still need good demo videos} but we should not hide the other
videos showing all the failure cases.

\subsection*{3) Robotics is expensive}

We often compare the unbelievable, exponential increase in computer
speed and capacity and the unbelievable, exponential decrease in price
of computing power to what will our robots be able to do once that
exponential growth kicks in.

Robots are not cheap and fast like computers, they are expensive and
fast or cheaper and slow or expensive and bulky, or cheap and cute...
robots are physical machines and it is impressive how we, humans,
quickly compare CPU power to electric motor power. Yes, large CPU
power has benefited robots massively and yes, both \emph{cost and
quality of robot parts are moving in the right direction}, but nobody
expects any other machine to improve exponentially. Why there is no
talk of a Moore's law for coffee machines? Or dishwashers? Or washing
machines? Or cars? Today's washing machines are extremely high quality
and extremely cheap, yet \emph{there is a limit to how cheap or how
powerful robots can get}.

Keep it real and never compare a robot to a phone or a laptop in terms
of price/performance, rather compare it to similar technologies such
as electric motors, air compressors, sensors, gearboxes, and transformers
(the alternate current ones, not the movie `transfrormers`, that is
science fiction)

Finally, The Isaac Asimov fans will be well aware that there are really
4 laws of robotics.

In the last of the \textquotedbl Foundation series\textquotedbl{}
books {[}https://en.wikipedia.org/wiki/Foundation\_and\_Earth {]}
Asimov introduces another law of robotics, called the zeroth law.

\subsection*{0) A robot may not harm humanity, or, by inaction, allow humanity
to come to harm.}

This one is good, \emph{this one is also for us to follow}, us the
people who build robots, so you can make sure your robot works in
a way that benefits all of humanity, you should give it a peaceful
and meaningful purpose to make this world a better place for us all.

{[}1{]} Robots will not do whatever we want them to do, because that's
not how software works. \emph{Your software will hardly ever do what
you want, but will faithfully do what you say}.
\end{document}
